\documentclass{article}
 
\usepackage{tikz}				% (note: AFTER \pdfminorversion)
\usetikzlibrary{arrows}
\usetikzlibrary{calc}
\usetikzlibrary{decorations.footprints}
\usetikzlibrary{decorations.pathmorphing}
\usetikzlibrary{decorations.text}
\usetikzlibrary{fadings}
\usetikzlibrary{fit}
\usetikzlibrary{matrix}
\usetikzlibrary{mindmap}
\usetikzlibrary{patterns}
\usetikzlibrary{positioning}
\usetikzlibrary{shadows}
\usetikzlibrary{shapes}
\usetikzlibrary{trees}

% NOTE FOR BEAMER USERS: Beamer messes around with catcodes. To solve the ``ampersand'' problem define the matrices as follows:
% 
% \matrix
% (nMatrix)
% [
% 	row sep					= 1cm,
% 	column sep				= 1cm,
% 	ampersand replacement	= \&
% ]
% {
% 	...
% };

\begin{document}


\tikzset
{
	BlocksStyle/.style =
	{
		% --------------------- shape properties ---------------------
		shape			= rectangle,			% shape
		rounded corners	= 0.0cm,				% radius of the rounded corner
		minimum height	= 0.7cm,				% | minimum size of the node
		minimum width	= 0.9cm,				% |
		rotate			= 0,					% angle of rotation
		scale			= 1.0,					% scaling factor
		%
		%
		% --------------------- border properties ---------------------
		draw			= black,				% draw the border with this color
		line width		= 0.02cm,				% thickness
		%
		%
		% --------------------- node filling properties ---------------------
		% possible choices:
		% 1 - transparent	= keep all commented
		% 2 - monocolored	= decomment only the ``fill'' line
		% 3 - shaded		= decomment only the ``top color'' and ``bottom color'' lines
%		fill			= red!10,				%
% 		top color		= white,				%
% 		bottom color	= white!10!black,		%
		%
		%
		% --------------------- text properties ---------------------
		% alignments: [flush left | left | flush center | center | flush right | right | justify]
		align			= center,				% text alignment
		text			= black,				% color of the fonts
		font			= \normalsize\normalfont,	% shape and dimension of the font
		inner xsep		= 0.2cm,				% min. dist. btw text and borders along x dimension
		inner ysep		= 0.2cm,				% min. dist. btw text and borders along x dimension
%		text width		= 0.5cm,				% max. width of the text
%		anchor			= base,					% text ``wobbling'' btw various nodes [center | base | mid]
		%
		%
		% --------------------- background image properties ---------------------
% 		path picture	=								%
% 		{\node at (path picture bounding box.center){	% [north | south | center | east | west]
% 			\includegraphics							%
% 			[height = 2.0cm, width = 1.0cm]				% stretching of the figure
% 			{../Images/logo_dei_small}};}				% file path
	}
}



\tikzset
{
	WideBlocksStyle/.style =
	{
		BlocksStyle,
		text width		= 2.0cm,				% max. width of the text
	}
}


\tikzset
{
	SumNodesStyle/.style =
	{
		% --------------------- shape properties ---------------------
		shape			= circle,				% shape
		minimum size	= 0.15cm,				% | minimum size of the node
		rotate			= 0,					% angle of rotation
		scale			= 1.0,					% scaling factor
		%
		%
		% --------------------- border properties ---------------------
		draw			= black,				% draw the border with this color
		line width		= 0.02cm,				% thickness
		%
		%
		% --------------------- node filling properties ---------------------
		% possible choices:
		% 1 - transparent	= keep all commented
		% 2 - monocolored	= decomment only the ``fill'' line
		% 3 - shaded		= decomment only the ``top color'' and ``bottom color'' lines
%		fill			= red!10,				%
% 		top color		= white,				%
% 		bottom color	= red!70!black,			%
		%
		%
		% --------------------- text properties ---------------------
		% alignments: [flush left | left | flush center | center | flush right | right | justify]
		align			= center,				% text alignment
		text			= black,				% color of the fonts
		font			= \normalsize\normalfont,	% shape and dimension of the font
% 		inner xsep		= 0.2cm,				% min. dist. btw text and borders along x dimension
% 		inner ysep		= 0.2cm,				% min. dist. btw text and borders along x dimension
% 		text width		= 2.0cm,				% max. width of the text
% 		anchor			= base,					% text ``wobbling'' btw various nodes [center | base | mid]
		%
		%
		% --------------------- background image properties ---------------------
% 		path picture	=								%
% 		{\node at (path picture bounding box.center){	% [north | south | center | east | west]
% 			\includegraphics							%
% 			[height = 2.0cm, width = 1.0cm]				% stretching of the figure
% 			{../Images/logo_dei_small}};}				% file path
	}
}



\tikzset
{
	HighlightingStyle/.style =
	{
		% --------------------- color properties ---------------------
		color				= black,	% color
		%
		%
		% --------------------- shape properties ---------------------
		line width			= 0.02cm,			% thickness
		arrows				= -,				% starting-ending arrows
% 		line cap			= round,			% line caps [rect | round | butt]
% 		line join			= round,			% how lines join [round | bevel | miter]
% 		rounded corners		= 0.0cm,			%
% 		in					= 0,				% starting angle (degrees)
% 		out					= 0,				% ending angle (degrees)
% 		shorten >			= 0.1cm,			% shorten the ending point
% 		shorten <			= 0.1cm,			% shorten the ending point
% 		double,									% make the line ``double''
% 		double distance		= 0.1cm,			% distance btw the two lines
		%
		%
		% --------------------- dashing properties ---------------------
		% [solid | dotted | densely dotted | loosely dotted | dashed | densely dashed | loosely dashed]
		dashed,
		% custom dashing:
% 		dash pattern		= on 0.1cm    off 0.1cm    on 0.2cm    off 0.2cm,
% 		dash phase			= 0.1cm,			% initial phase
		%
		%
		% --------------------- decoration properties ---------------------
% 		decorate,						%
% 		decoration	=					% |
% 		{								% |
% 		}								% |
	}
}	% no semicolons are here required

\input{LinesStyle}
%

\begin{figure}[!htb]
\begin{center}
\begin{tikzpicture}
[
	xscale	= 1,	% to scale horizontally everything but the text
	yscale	= 1,	% to scale vertically everything but the text
]


% ------------------------------------------------------
% NODES DEFINITION
\matrix
[
	row sep		= 1cm,
	column sep	= 1cm,
]
{
	% ------------------------------ row 1
	\node (nInput) {$u$}; &
	\node (nSystem) [BlocksStyle] {system}; &
	\node (nOutput) {$y$}; \\
};



% ------------------------------------------------------
% PATHS
\draw [LinesStyle] (nInput) -- (nSystem);
\draw [LinesStyle] (nSystem) -- (nOutput);


\end{tikzpicture}
%
\caption{1.}
\label{fig:1}
%
\end{center}
\end{figure}

\clearpage

\begin{figure}[!htb]
\begin{center}
\begin{tikzpicture}
[
	xscale	= 1,	% to scale horizontally everything but the text
	yscale	= 1,	% to scale vertically everything but the text
]


% ------------------------------------------------------
% NODES DEFINITION
\matrix
[
	row sep		= 1cm,
	column sep	= 1cm,
]
{
	% ------------------------------ row 1
	&
	\node (nInputError) {$u^{er}$}; &
	&
	\node (nOutputError) {$y^{er}$}; &
	\\
	%
	%
	% ------------------------------ row 2
	\node (nInput) {$u$}; &
	\node (nLeftSum) [SumNodesStyle] {}; &
	\node (nSystem) [BlocksStyle] {Jacobian \\ linearizator}; &
	\node (nRightSum) [SumNodesStyle] {}; &
	\node (nOutput) {$y$}; \\
};



% ------------------------------------------------------
% PATHS
\draw [LinesStyle] (nInput) -- (nLeftSum) node [near end, above] {$+$};
\draw [LinesStyle] (nInputError) -- (nLeftSum) node [near end, left] {$-$};
\draw [LinesStyle] (nLeftSum) -- (nSystem) node [midway, above] {$\tilde{u}$};
\draw [LinesStyle] (nSystem) -- (nRightSum)
	node [midway, above] {$\tilde{y}$}
	node [near end, above] {$+$}; % the last semicolon closes the initial \draw
\draw [LinesStyle] (nRightSum) -- (nOutput);
\draw [LinesStyle] (nOutputError) -- (nRightSum) node [near end, left] {$-$};



\end{tikzpicture}
%
\caption{2.}
\label{fig:2}
%
\end{center}
\end{figure}

\clearpage

\begin{figure}[!htb]
\begin{center}
\begin{tikzpicture}
[
	xscale	= 1,	% to scale horizontally everything but the text
	yscale	= 1,	% to scale vertically everything but the text
]


% ------------------------------------------------------
% NODES DEFINITION
\matrix
[
	row sep		= 1cm,
	column sep	= 1cm,
]
{
	% ------------------------------ row 1
	&
	&
	\node (nOutputError) {$\tilde{y}$}; &
	\\
	%
	%
	% ------------------------------ row 2
	\node (nInput) {$\tilde{u}$}; &
	\node (nSystem) [BlocksStyle] {controller}; &
	\node (nSum) [SumNodesStyle] {}; &
	\node (nOutput) {$\tilde{r}$}; \\
};



% ------------------------------------------------------
% PATHS
\draw [LinesStyle] (nOutput) -- (nSum) node [near end, above] {$+$};
\draw [LinesStyle] (nOutputError) -- (nSum) node [near end, right] {$-$};
\draw [LinesStyle] (nSum) -- (nSystem);
\draw [LinesStyle] (nSystem) -- (nInput);



\end{tikzpicture}
%
\caption{3.}
\label{fig:3}
%
\end{center}
\end{figure}

\clearpage

\begin{figure}[!htb]
\begin{center}
\begin{tikzpicture}
[
	xscale	= 1,	% to scale horizontally everything but the text
	yscale	= 1,	% to scale vertically everything but the text
]


% ------------------------------------------------------
% NODES DEFINITION
\matrix
[
	row sep		= 1cm,
	column sep	= 1cm,
]
{
	% ------------------------------ row 1
	&
	\node (nInputError) {$u^{er}$}; &
	&
	\node (nOutputError) {$y$}; &
	\\
	%
	%
	% ------------------------------ row 2
	\node (nInput) {$\tilde{u}$}; &
	\node (nLeftSum) [SumNodesStyle] {}; &
	\node (nSystem) [BlocksStyle] {controller}; &
	\node (nRightSum) [SumNodesStyle] {}; &
	\node (nOutput) {$\tilde{r}$}; \\
};



% ------------------------------------------------------
% PATHS
\draw [LinesStyle] (nOutput) -- (nRightSum) node [near end, above] {$+$};
\draw [LinesStyle] (nOutputError) -- (nRightSum) node [near end, right] {$-$};
\draw [LinesStyle] (nRightSum) -- (nSystem);
\draw [LinesStyle] (nSystem) -- (nLeftSum)
	node [midway, above] {$\tilde{u}$}
	node [near end, above] {$+$};
\draw [LinesStyle] (nInputError) -- (nLeftSum) node [near end, right] {$+$};
\draw [LinesStyle] (nLeftSum) -- (nInput);



\end{tikzpicture}
%
\caption{4.}
\label{fig:4}
%
\end{center}
\end{figure}

\clearpage

\begin{figure}[!htb]
\begin{center}
\begin{tikzpicture}
[
	xscale	= 1,	% to scale horizontally everything but the text
	yscale	= 1,	% to scale vertically everything but the text
]



% ------------------------------------------------------
% NODES DEFINITION
\matrix
(nMatrix)
[
	row sep		= 1cm,
	column sep	= 1cm,
]
{
	% ------------------------------ row 1
	\node (nInput) {$u$}; &
	\node (nLeftSum) [SumNodesStyle] {}; &
	\node (nSystem) [BlocksStyle] {system}; &
	\node (nM14) {}; &
	\node (nOutput) {$y$}; \\
	%
	%
	% ------------------------------ row 2
	\node (nM21) {}; &
	\node (nM22) {}; &
	\node (nController) [BlocksStyle] {controller}; &
	\node (nRightSum) [SumNodesStyle] {}; &
	\node (nReference) {$r$}; \\
};



% ------------------------------------------------------
% PATHS
\draw [LinesStyle] (nInput) -- (nLeftSum) node [near end, below] {$+$};
\draw [LinesStyle] (nLeftSum) -- (nSystem);
\draw [LinesStyle] (nSystem) -- (nOutput);
\draw [LinesStyle] (nSystem) -| (nRightSum) node [at end, above right] {$+$};
\draw [LinesStyle] (nReference) -- (nRightSum) node [near end, above] {$+$};
\draw [LinesStyle] (nRightSum) -- (nController);
\draw [LinesStyle] (nController) -| (nLeftSum) node [at end, below left] {$+$};




% -------------------------
% auxiliary nodes
\node [coordinate, xshift = -0.5cm, yshift =  0.5cm] (nAux1) at (nInput.north west) {};
\node [coordinate, xshift =  0.5cm, yshift =  0.5cm] (nAux2) at (nLeftSum.north east) {};
\node [coordinate, xshift =  0.5cm, yshift =  0.5cm] (nAux3) at (nRightSum.north east) {};
\node [coordinate, xshift =  0.5cm, yshift = -0.5cm] (nAux4) at (nRightSum.south east) {};
%
\draw [HighlightingStyle] (nAux1) -| (nAux2) |- (nAux3) -- (nAux4) -| (nAux1)
node [above, pos = 0.43] {controller};




\end{tikzpicture}
%
\caption{5.}
\label{fig:5}
%
\end{center}
\end{figure}

\clearpage

\begin{figure}[!htb]
\begin{center}
\begin{tikzpicture}
[
	xscale	= 1,	% to scale horizontally everything but the text
	yscale	= 1,	% to scale vertically everything but the text
]


% ------------------------------------------------------
% NODES DEFINITION
\matrix
[
	row sep		= 1cm,
	column sep	= 1cm,
]
{
	% ------------------------------ row 1
	&
	&
	\node (nSystem) [BlocksStyle] {NL Plant}; &
	\node [coordinate] (nM14) {}; &
	\node (nOutput) {$y$}; \\
	%
	%
	% ------------------------------ row 2
	\node (nLeftSum) [SumNodesStyle] {}; &
	\node (nError) [BlocksStyle] {$u^{eq}(y)$}; &
	&
	\node [coordinate, xshift = -0.7cm] (nM24) {}; &
	\\
	%
	%
	% ------------------------------ row 3
	&
	&
	\node (nController) [BlocksStyle] {$K(y)$}; &
	\node (nRightSum) [SumNodesStyle] {}; &
	\node (nReference) {$r$}; \\
};



% ------------------------------------------------------
% PATHS
\draw [LinesStyle] (nLeftSum) |- (nSystem);
\draw [LinesStyle] (nSystem) -- (nOutput);
\draw [LinesStyle] (nError) -- (nLeftSum) node [near end, above] {$+$};
\draw [LinesStyle] (nController) -| (nLeftSum) node [at end, below right] {$+$};
\draw [LinesStyle] (nM14) -- (nRightSum) node [at end, above left] {$-$};
\draw [LinesStyle] (nM14) |- (nError);
\draw [LinesStyle] (nReference) -- (nRightSum) node [at end, above right] {$+$};
\draw [LinesStyle] (nRightSum) -- (nController);
\draw [LinesStyle] (nM24) |- (nController.15);




% -------------------------
% auxiliary nodes
\node [coordinate, xshift = -0.5cm, yshift =  0.5cm] (nAux1) at (nLeftSum.north west) {};
\node [coordinate, xshift =  0.7cm, yshift = -0.5cm] (nAux2) at (nRightSum.south east) {};
%
\draw [HighlightingStyle] (nAux1) -| (nAux2) -| (nAux1)
node [above, pos = 0.43] {controller};




\end{tikzpicture}
%
\caption{6.}
\label{fig:6}
%
\end{center}
\end{figure}

\clearpage

\begin{figure}[!htb]
\begin{center}
\begin{tikzpicture}
[
	xscale	= 1,	% to scale horizontally everything but the text
	yscale	= 1,	% to scale vertically everything but the text
]


% ------------------------------------------------------
% NODES DEFINITION
\matrix
[
	row sep		= 1cm,
	column sep	= 1cm,
]
{
	% ------------------------------ row 1
	\node (nInput) {$v$}; &
	\node (nSum) [SumNodesStyle] {}; &
	&
	\node (nSystem) [BlocksStyle] {system}; &
	\node (nOutput) {$y$}; \\
	%
	%
	% ------------------------------ row 2
	&
	&
	\node (nController) [BlocksStyle] {$L$}; &
	&
	\\
};



% ------------------------------------------------------
% PATHS
\draw [LinesStyle] (nInput) -- (nSum) node [near end, above] {$+$};
\draw [LinesStyle] (nSum) -- (nSystem) node [midway, above] {$u$};
\draw [LinesStyle] (nController) -| (nSum) node [at end, below left] {$+$};
\draw [LinesStyle] (nSystem) -- (nOutput);
\draw [LinesStyle, dotted] (nSystem) |- (nController) node [near end, above] {$x$};


\end{tikzpicture}
%
\caption{7.}
\label{fig:7}
%
\end{center}
\end{figure}

\clearpage

\begin{figure}[!htb]
\begin{center}
\begin{tikzpicture}
[
	xscale	= 1,	% to scale horizontally everything but the text
	yscale	= 1,	% to scale vertically everything but the text
]



% ------------------------------------------------------
% NODES DEFINITION
\matrix
(nMatrix)
[
	row sep		= 1cm,
	column sep	= 1cm,
]
{
	% ------------------------------ row 1
	\node (nInput) {$v$}; &
	\node (nSum) [SumNodesStyle] {}; &
	\node (nM13) [coordinate] {}; &
	\node (nSystem) [BlocksStyle] {system}; &
	\node (nM15) {}; &
	\node (nOutput) {$y$}; \\
	%
	%
	% ------------------------------ row 2
	\node (nM21) [coordinate] {}; &
	\node (nM22) [coordinate] {}; &
	\node (nM23) [coordinate] {}; &
	\node (nM24) [coordinate] {}; &
	\node (nM25) [coordinate] {}; &
	\node (nM26) [coordinate] {}; \\
	%
	%
	% ------------------------------ row 3
	\node (nM31) [coordinate] {}; &
	\node (nM32) [coordinate] {}; &
	\node (nController) [BlocksStyle] {L}; &
	\node (nObserver) [BlocksStyle] {observer}; &
	\node (nM35) [coordinate] {}; &
	\\
};



% ------------------------------------------------------
% PATHS
\draw [LinesStyle] (nInput) -- (nSum) node [near end, below] {$+$};
\draw [LinesStyle] (nSum) -- (nSystem) node [midway, above] {$u$};
\draw [LinesStyle] (nSystem) -- (nOutput);
\draw [LinesStyle, -] (nSystem) -| (nM35);
\draw [LinesStyle] (nM35) -- (nObserver);
\draw [LinesStyle] (nObserver) -- (nController) node [midway, above] {$\hat{x}$};
\draw [LinesStyle] (nController) -| (nSum) node [at end, below left] {$+$};
\draw [LinesStyle, -] (nM13) -- (nM23);
\draw [LinesStyle] (nM23) -| (nObserver);




% -------------------------
% auxiliary nodes
\node [coordinate, xshift = -0.5cm, yshift =  0.5cm] (nAux1) at (nSum.north west) {};
\node [coordinate, xshift =  0.5cm, yshift =  0.5cm] (nAux2) at (nM23.north east) {};
\node [coordinate, xshift =  0.5cm, yshift = -1.0cm] (nAux3) at (nM35.south east) {};
%
\draw [HighlightingStyle] (nAux1) -| (nAux2) -| (nAux3) -| (nAux1)
node [above, pos = 0.43] {controller};




\end{tikzpicture}
%
\caption{8.}
\label{fig:8}
%
\end{center}
\end{figure}

\clearpage

\begin{figure}[!htb]
\begin{center}
\begin{tikzpicture}
[
	xscale	= 1,	% to scale horizontally everything but the text
	yscale	= 1,	% to scale vertically everything but the text
]



% ------------------------------------------------------
% NODES DEFINITION
\matrix
(nMatrix)
[
	row sep		= 1cm,
	column sep	= 1cm,
]
{
	% ------------------------------ row 1
	\node (nInput) {$v$}; &
	\node (nLeftSum) [SumNodesStyle] {}; &
	\node (nSystem) [BlocksStyle] {plant}; &
	\node (nM14) {}; &
	\node (nOutput) {$y$}; \\
	%
	%
	% ------------------------------ row 2
	\node (nM21) {}; &
	\node (nM22) {}; &
	\node (nController) [BlocksStyle] {controller}; &
	\node (nRightSum) [SumNodesStyle] {}; &
	\node (nReference) {$r$}; \\
};



% ------------------------------------------------------
% PATHS
\draw [LinesStyle] (nInput) -- (nLeftSum) node [near end, below] {$+$};
\draw [LinesStyle] (nLeftSum) -- (nSystem) node [midway, above] {$u$};
\draw [LinesStyle] (nSystem) -- (nOutput);
\draw [LinesStyle] (nSystem) -| (nRightSum) node [at end, above right] {$+$};
\draw [LinesStyle, dotted] (nReference) -- (nRightSum) node [near end, above] {$+$};
\draw [LinesStyle] (nRightSum) -- (nController);
\draw [LinesStyle] (nController) -| (nLeftSum) node [at end, below left] {$+$};






\end{tikzpicture}
%
\caption{9.}
\label{fig:9}
%
\end{center}
\end{figure}

\clearpage

\begin{figure}[!htb]
\begin{center}
\begin{tikzpicture}
[
	xscale	= 1,	% to scale horizontally everything but the text
	yscale	= 1,	% to scale vertically everything but the text
]



% ------------------------------------------------------
% NODES DEFINITION
\matrix
(nMatrix)
[
	row sep		= 1cm,
	column sep	= 1cm,
]
{
	% ------------------------------ row 1
	\node (nM11) [coordinate] {}; &
	\node (nSystem) [BlocksStyle] {plant}; &
	\node (nM14) {}; &
	\node (nOutput) {$y$}; \\
	%
	%
	% ------------------------------ row 2
	\node (nM21) {}; &
	\node (nController) [BlocksStyle] {controller}; &
	\node (nSum) [SumNodesStyle] {}; &
	\node (nReference) {$r$}; \\
};



% ------------------------------------------------------
% PATHS
\draw [LinesStyle] (nM11) -- (nSystem) node [midway, above] {$u$};
\draw [LinesStyle] (nSystem) -- (nOutput);
\draw [LinesStyle] (nSystem) -| (nSum) node [at end, above right] {$-$};
\draw [LinesStyle] (nReference) -- (nSum) node [near end, above] {$+$};
\draw [LinesStyle] (nSum) -- (nController);
\draw [LinesStyle, -] (nController) -| (nM11);




\end{tikzpicture}
%
\caption{10.}
\label{fig:10}
%
\end{center}
\end{figure}

\clearpage

\begin{figure}[!htb]
\begin{center}
\begin{tikzpicture}
[
	xscale	= 1,	% to scale horizontally everything but the text
	yscale	= 1,	% to scale vertically everything but the text
]



% ------------------------------------------------------
% NODES DEFINITION
\matrix
(nMatrix)
[
	row sep		= 1cm,
	column sep	= 1cm,
]
{
	% ------------------------------ row 1
	\node (nInput) {$u$}; &
	\node (nSystem) [BlocksStyle] {plant}; &
	\node (nM14) {}; &
	\node (nOutput) {$y$}; \\
	%
	%
	% ------------------------------ row 2
	\node (nM21) {}; &
	\node (nController) [BlocksStyle] {controller}; &
	\node (nSum) [SumNodesStyle] {}; &
	\node (nReference) {$r$}; \\
};



% ------------------------------------------------------
% PATHS
\draw [LinesStyle] (nInput) -- (nSystem);
\draw [LinesStyle] (nSystem) -- (nOutput);
\draw [LinesStyle] (nSystem) -| (nSum) node [at end, above right] {$-$};
\draw [LinesStyle] (nReference) -- (nSum) node [near end, above] {$+$};
\draw [LinesStyle] (nSum) -- (nController);
\draw [LinesStyle] (nController) -- (nM21);




% -------------------------
% auxiliary nodes
\node [coordinate, xshift = -0.5cm, yshift =  0.5cm] (nAux1) at (nSystem.north west) {};
\node [coordinate, xshift =  0.5cm, yshift = -1.0cm] (nAux2) at (nReference.south east) {};
%
\draw [HighlightingStyle] (nAux1) -| (nAux2) -| (nAux1)
node [above, pos = 0.38] {augmented plant};




\end{tikzpicture}
%
\caption{11.}
\label{fig:11}
%
\end{center}
\end{figure}

\clearpage

\begin{figure}[!htb]
\begin{center}
\begin{tikzpicture}
[
	xscale	= 1,	% to scale horizontally everything but the text
	yscale	= 1,	% to scale vertically everything but the text
]



% ------------------------------------------------------
% NODES DEFINITION
\matrix
(nMatrix)
[
	row sep		= 1cm,
	column sep	= 1cm,
]
{
	% ------------------------------ row 1
	\node (nM11) [coordinate] {}; &
	\node (nM12) [coordinate] {}; &
	\node (nSystem) [BlocksStyle] {plant}; &
	\node (nM15) {}; &
	\node (nOutput) {$y$}; \\
	%
	%
	% ------------------------------ row 2
	\node (nM21) {}; &
	\node (nController) [BlocksStyle] {$K(s)$}; &
	\node (nIntegrator) [BlocksStyle] {$\int $}; &
	\node (nSum) [SumNodesStyle] {}; &
	\node (nReference) {$r$}; \\
};



% ------------------------------------------------------
% PATHS
\draw [LinesStyle] (nM11) -- (nSystem) node [midway, above] {$u$};
\draw [LinesStyle] (nSystem) -- (nOutput);
\draw [LinesStyle] (nSystem) -| (nSum) node [at end, above right] {$-$};
\draw [LinesStyle] (nReference) -- (nSum) node [near end, above] {$+$};
\draw [LinesStyle] (nSum) -- (nIntegrator);
\draw [LinesStyle] (nIntegrator) -- (nController);
\draw [LinesStyle, -] (nController) -| (nM11);



% -------------------------
% auxiliary nodes
\node [coordinate, xshift = -0.5cm, yshift =  0.5cm] (nAux1) at (nController.north west) {};
\node [coordinate, xshift =  0.5cm, yshift = -0.5cm] (nAux2) at (nIntegrator.south east) {};
%
\draw [HighlightingStyle] (nAux1) -| (nAux2) -| (nAux1)
node [above, pos = 0.43] {controller};





\end{tikzpicture}
%
\caption{12.}
\label{fig:12}
%
\end{center}
\end{figure}

\clearpage

\begin{figure}[!htb]
\begin{center}
\begin{tikzpicture}
[
	xscale	= 1,	% to scale horizontally everything but the text
	yscale	= 1,	% to scale vertically everything but the text
]


% ------------------------------------------------------
% NODES DEFINITION
\matrix
[
	row sep		= 1cm,
	column sep	= 1cm,
]
{
	% ------------------------------ row 1
	\node (nInput) {$u$}; &
	\node (nM12) [coordinate] {}; &
	\node (nSystem) [BlocksStyle] {plant}; &
	&
	\node (nM15) [coordinate] {}; &
	\\
	%
	%
	% ------------------------------ row 2
	&
	&
	&
	\node (nM24) [coordinate, xshift = 1cm] {}; &
	\node (nSum) [SumNodesStyle] {}; &
	\\
	%
	%
	% ------------------------------ row 3
	&
	&
	&
	\node (nSystemCopy) [BlocksStyle] {copy of \\ plant}; &
	&
	\\
	%
	%
	% ------------------------------ row 4
	&
	&
	&
	\node (nM44) [coordinate, xshift = -1cm] {}; &
	&
	\node (nState) {$\hat{x}$}; \\
};



% ------------------------------------------------------
% PATHS
\draw [LinesStyle] (nInput) -- (nSystem);
\draw [LinesStyle] (nSystem) -| (nSum)
	node [at end, above right] {$+$}
	node [near start, above] {$y$};
\draw [LinesStyle] (nM12) |- (nSystemCopy);
\draw [LinesStyle] (nSystemCopy) -| (nSum)
	node [at end, below right] {$-$}
	node [near start, above] {$\hat{y}$};
\draw [LinesStyle] (nSystemCopy) |- (nState);
\draw [LinesStyle, -] (nSum) -- (nM24);
\draw [LinesStyle] (nM24) -- (nM44);




% -------------------------
% auxiliary nodes
\node [coordinate, xshift =  0.5cm, yshift =  0.5cm] (nAux1) at (nSum.north east) {};
\node [coordinate, xshift = -0.5cm, yshift = -0.5cm] (nAux2) at (nM44.south west) {};
%
\draw [HighlightingStyle] (nAux1) -| (nAux2) -| (nAux1)
node [above, pos = 0.40] {observer};




\end{tikzpicture}
%
\caption{13.}
\label{fig:13}
%
\end{center}
\end{figure}

\clearpage

\begin{figure}[!htb]
\begin{center}
\begin{tikzpicture}
[
	xscale	= 1,	% to scale horizontally everything but the text
	yscale	= 1,	% to scale vertically everything but the text
]


% ------------------------------------------------------
% NODES DEFINITION
\matrix
[
	row sep		= 1cm,
	column sep	= 1cm,
]
{
	% ------------------------------ row 1
	\node (nInput) {$u$}; &
	\node (nM12) [coordinate] {}; &
	\node (nSystem) [BlocksStyle] {system}; &
	&
	&
	&
	&
	&
	\\
	%
	%
	% ------------------------------ row 2
	&
	&
	&
	&
	&
	&
	\node (nK) [BlocksStyle] {$K$}; &
	\node (nRightSum) [SumNodesStyle] {}; &
	\\
	%
	%
	% ------------------------------ row 3
	&
	&
	\node (nB) [BlocksStyle] {$B$}; &
	\node (nLeftSum) [SumNodesStyle] {}; &
	\node (nIntegrator) [BlocksStyle] {$\int $}; &
	\node (nM36) [coordinate] {}; &
	\node (nC) [BlocksStyle] {$C$}; &
	&
	\\
	%
	%
	% ------------------------------ row 4
	&
	&
	\node (nM43) [coordinate] {}; &
	&
	\node (nA) [BlocksStyle] {$A$}; &
	&
	&
	&
	\node (nState) {$\hat{x}$}; \\
};



% ------------------------------------------------------
% PATHS
\draw [LinesStyle] (nInput) -- (nSystem);
\draw [LinesStyle] (nSystem) -| (nRightSum)
	node [at end, above right] {$+$}
	node [midway, above] {$y$};
\draw [LinesStyle] (nRightSum) -- (nK);
\draw [LinesStyle] (nK) -| (nLeftSum) node [at end, above left] {$+$};
\draw [LinesStyle] (nM12) |- (nB);
\draw [LinesStyle] (nB) |- (nLeftSum) node [at end, below left] {$+$};
\draw [LinesStyle] (nLeftSum) -- (nIntegrator);
\draw [LinesStyle] (nIntegrator) -- (nC) node [midway, above] {$\hat{x}$};
\draw [LinesStyle] (nC) -| (nRightSum)
	node [at end, below right] {$-$}
	node [midway, right] {$\hat{y}$};
\draw [LinesStyle] (nA) -| (nLeftSum) node [at end, below right] {$+$};
\draw [LinesStyle] (nM36) |- (nState);
\draw [LinesStyle] (nM36) |- (nA);




% -------------------------
% auxiliary nodes
\node [coordinate, xshift =  0.5cm, yshift =  0.5cm] (nAux1) at (nRightSum.north east) {};
\node [coordinate, xshift = -1.5cm, yshift = -0.8cm] (nAux2) at (nM43.south west) {};
%
\draw [HighlightingStyle] (nAux1) -| (nAux2) -| (nAux1)
node [above, pos = 0.43] {observer};




\end{tikzpicture}
%
\caption{14.}
\label{fig:14}
%
\end{center}
\end{figure}

\clearpage
\begin{figure}[!htb]
\begin{center}
\begin{tikzpicture}
[
	xscale	= 1,	% to scale horizontally everything but the text
	yscale	= 1,	% to scale vertically everything but the text
]

% ------------------------------------------------------
% NODES DEFINITION
\matrix
(nMatrix)
[
	row sep		= 1cm,
	column sep	= 1cm,
]
{
	% ------------------------------ row 1
	\node (nInput) {$v$}; &
	\node (nLeftSum) [SumNodesStyle] {}; &
	\node (nSystem) [BlocksStyle] {plant}; &
	\node (nM14) {}; &
	\node (nOutput) {$y$}; \\
	%
	%
	% ------------------------------ row 2
	\node (nM21) {}; &
	\node (nM22) {}; &
	\node (nController) [BlocksStyle] {controller}; &
	\node (nRightSum) [coordinate] {}; &
	\\
};

% ------------------------------------------------------
% PATHS
\draw [LinesStyle] (nInput) -- (nLeftSum) node [near end, below] {$+$};
\draw [LinesStyle] (nLeftSum) -- (nSystem) node [midway, above] {$u$};
\draw [LinesStyle] (nSystem) -- (nOutput);
\draw [LinesStyle, -] (nSystem) -| (nRightSum);
\draw [LinesStyle] (nRightSum) -- (nController);
\draw [LinesStyle] (nController) -| (nLeftSum) node [at end, below left] {$-$};

\end{tikzpicture}
%
\caption{15.}
\label{fig:15}
%
\end{center}
\end{figure}

\clearpage
\input{Figure__16}
\clearpage
\begin{figure}[!htb]
\begin{center}
\begin{tikzpicture}
[
	xscale	= 1,	% to scale horizontally everything but the text
	yscale	= 1,	% to scale vertically everything but the text
]
	%
	\node (u)													{$u(t)$};
	\node (system)		[BlocksStyle, right = 1cm of u]			{$g(k, \tau)$};
	\node (samplerA)	[right = 1cm of system, coordinate]		{};
	\node (samplerB)	[right = 0.5cm of samplerA, coordinate] {};
	\node (samplerC)	[above = 0.5cm of samplerB, coordinate] {};
	\node (sum)			[circle, draw, minimum size = 0.4cm, right = 1cm of samplerB] {+};
	\node (y)			[right = 1cm of sum]					{$y(k)$};
	\node (v)			[above = 0.5cm of sum]					{$v(k)$};
	%
	\draw [LinesStyle] (u) -- (system);
	\draw [LinesStyle, -] (system) -- (samplerA);
	\draw [LinesStyle] (samplerA) -- (samplerC);
	\draw [LinesStyle, -, dotted] (samplerB) -- (samplerC);
	\draw [LinesStyle] (samplerB) -- (sum);
	\draw [LinesStyle] (v) -- (sum);
	\draw [LinesStyle] (sum) -- (y);
	%
\end{tikzpicture}
%
\caption{17.}
\label{fig:17}
%
\end{center}
\end{figure}

\clearpage
\input{Figure__18}
\clearpage

\begin{figure}[!htb]
\begin{center}
\begin{tikzpicture}
[
	xscale	= 1,	% to scale horizontally everything but the text
	yscale	= 1,	% to scale vertically everything but the text
]


% ------------------------------------------------------
% NODES DEFINITION
\matrix
[
	row sep		= 1cm,
	column sep	= 1cm,
]
{
	% ------------------------------ row 1
	\node (nEvent) {$e_{t}$}; &
	\node (nInput) [BlocksStyle] {I}; &
	\node (nPlant) [BlocksStyle] {P}; &
	\node (nOutput) {$y_{t}$}; \\
	%
	% ------------------------------ row 2
	&
	&
	&
	\node (nFeatures) {$f_{t}$}; \\
};
%
\node (nMeasuredInput) [above = 1cm of nPlant.120] {$u_{t}$};
\node (nDisturbance) [above = 1cm of nPlant.60] {$d_{t}$};


% ------------------------------------------------------
% PATHS
\draw [LinesStyle] (nEvent) -- (nInput);
\draw [LinesStyle] (nInput) -- (nPlant) node [midway, above] {$o_{t}$};
\draw [LinesStyle] (nPlant) -- (nOutput);
\draw [LinesStyle] (nInput) |- (nFeatures);
\draw [LinesStyle] (nMeasuredInput) -- (nPlant.120);
\draw [LinesStyle] (nDisturbance) -- (nPlant.60);



\end{tikzpicture}
%
\caption{19.}
\label{fig:19}
%
\end{center}
\end{figure}


\clearpage

\begin{figure}[!htb]
\begin{center}
\begin{tikzpicture}
[
	xscale	= 1,	% to scale horizontally everything but the text
	yscale	= 1,	% to scale vertically everything but the text
]


% ------------------------------------------------------
% NODES DEFINITION
\matrix
[
	row sep		= 1cm,
	column sep	= 1cm,
]
{
	% ------------------------------ row 1
	\node (nM11) [] {$\alpha_{ref}$}; &
	\node (nM12) [SumNodesStyle] {}; &
	\node (nM13) [BlocksStyle] {P}; &
	\node (nM14) [SumNodesStyle] {}; &
	\node (nM15) [BlocksStyle] {PI}; &
	\node (nM16) [BlocksStyle] {$\displaystyle \frac{\beta}{\tau s + 1}$}; &
	\node (nM17) [BlocksStyle] {$\int$}; &
	\node (nM18) [] {}; \\
	% ------------------------------ row 2
	\node (nM21) [coordinate] {}; &
	\node (nM22) [coordinate] {}; &
	\node (nM23) [coordinate] {}; &
	\node (nM24) [coordinate] {}; &
	\node (nM25) [coordinate] {}; &
	\node (nM26) [coordinate] {}; &
	\node (nM27) [coordinate] {}; &
	\node (nM28) [coordinate] {}; \\
	% ------------------------------ row 3
	\node (nM31) [coordinate] {}; &
	\node (nM32) [coordinate] {}; &
	\node (nM33) [coordinate] {}; &
	\node (nM34) [coordinate] {}; &
	\node (nM35) [coordinate] {}; &
	\node (nM36) [coordinate] {}; &
	\node (nM37) [coordinate] {}; &
	\node (nM38) [coordinate] {}; \\
};



% ------------------------------------------------------
% PATHS
\draw [LinesStyle] (nM11) -- (nM12)
	node [near end, below] {$+$};
\draw [LinesStyle] (nM12) -- (nM13)
	node [midway, above] {$e_{att}$};
\draw [LinesStyle] (nM13) -- (nM14)
	node [near end, below] {$+$};
\draw [LinesStyle] (nM14) -- (nM15)
	node [midway, above] {$e_{rate}$};
\draw [LinesStyle] (nM15) -- (nM16);
\draw [LinesStyle] (nM16) -- (nM17)
	node [midway, above] (nAux1) {$\omega$}
	node [coordinate, midway, fill, circle, inner sep=0pt, minimum size=3pt] {};
\draw [color=black,line width=0.02cm,solid] (nAux1) |- (nM24);
\draw [LinesStyle] (nM24) -- (nM14)
	node [near end, left] {$-$};
\draw [LinesStyle] (nM17) -- (nM18)
	node [midway, above] (nAux2) {$\alpha$}
	node [coordinate, midway, fill, circle, inner sep=0pt, minimum size=3pt] {};
\draw [color=black,line width=0.02cm,solid] (nAux2) |- (nM32);
\draw [LinesStyle] (nM32) -- (nM12)
	node [very near end, left] {$-$};


\end{tikzpicture}
%
\caption{20.}
\label{fig:20}
%
\end{center}
\end{figure}

\clearpage

\end{document}
